\chapter{Proposed Approach}
\label{chapter:approach}




\section{Introduction}
\label{section:ApproachIntroduction}
As introduced in the state of the art section,  DDoS attacks are based on multiple flows strategies. As presented in Figure \ref{fig:unveilingg}, each attacker sends a large amount of malicious flows which will be received by the victim. Hereby, a DDoS attack can be modeled through 1) one or more sources (attackers), 2) one or more destinations (victims), and 3) a set of attack flows. In other words, in DDoS attacks, as one or many sources sends malicious flows to one or many destinations, a key idea is that flows of the same attacks can have the same source or the same destination. As a result, relying a flow with other flows with flows of the same attacks can be achieved through its source and destination.\\


\section{Background}

\section{Overview}
\label{overviw}
 Our approach is loosely similar to \cite{unveiling}, we use but differs from it into the following points: 1) we use a simpler mechanism to aggregate messages and updates nodes states. 2) It requires only one message-passing iteration. for each flow f, 3) A flow f final state $z_f$ combines features of the flow as well as features obtained from attackers and victims (the presented points 1, 2 and 3 in the introduction section \ref{section:ApproachIntroduction}). This combination make final flow's state with better expressiveness and more suitable for classification. Each flow in the DDoS attack is associated to a source which is an attacker and a destination which is a victim. So, the classification of a flow requires the consideration of both the source and the destination in the flow's features.  \\
Therefore, each flow will be defined through its specific features to which features of the source and the destination are added.  
The flow's source $S_f$, can be extracted from the sent flow and the destination $D_f$ state can be extracted from he received flows. In fact, $z_f$ implicitly expresses the flows having $S_f$ as source and the flows having $S_f$ as destination. Graph Neural Networks (GNN) can achieve this objective by exploiting its message-passing abilities to compute $z_f$. The\textcolor{white}{.} proposed\textcolor{white}{.} approach is\textcolor{white}{.} illustrated\textcolor{white}{.} in\textcolor{white}{.} figure \ref{fig:app_fig}.

\begin{figure}[H]
    \centering
    \includegraphics[scale=0.85]{figures/approach_gnn_flow_chart.png}
    \captionsetup{font=large}
    \caption{Approach Overview}
        \label{fig:app_fig}
\end{figure}


\section{Graph Representation}
In this section we provide a detailed introduction for the proposed graph representation. 
One major contribution in our approach is introducing a new graph encoding for the communication graph. It consists of modifying the communication graph proposed in \cite{unveiling} in order to obtain a suitable graph encoding for an optimized message passing mechanism and achieve better expressive nodes hidden states.  
\subsection{Notations}
This subsection is dedictes to explain the meaning of some notations related to the proposed approach. This will ease the comprehension of the next parts. 
A network consists of a graph $G =(V, E)$ where V the set of nodes and E represents the set of edges. \ref{table:notations} presents a list of adopted notations.
\begin{table}[H]
\centering
\begin{tabular}{|c|c|}
\hline
Notations & Meaning \\
\hline
$E_t$ & The sub set of $E$ that contains all edges of type $t$\\
\hline
$h_i$ & A nodes representing a host $i$\\
\hline
$f_i$ & A nodes representing a flow $i$\\
\hline
$Xh_i$ & The hidden state of the node $h_i \text{ represneting the host } i$\\
\hline
$Xv_i$ & The hidden state of the node $f_i \text{ represneting the flow } i$\\
\hline
$x_v$ & Can be any sub-state of a node $v \in V$\\
\hline
$z_{f_i}$ & The final state of the flow $i$ \\
\hline
\end{tabular}
\caption{Notations}
\label{table:notations}
\end{table}

\subsection{Graph Definition}
We define formally our graph encoding. A network is encoded in a Directed Multipartite (\ref{heterog} ) graph $G = (V, E)$, where: 
\begin{itemize}
    \item V = $V_{flows}  \cup  V_{hosts}$ the set of graph nodes,
    \item E = $\{ E_{fsrc}  \cup  E_{fdst}  \cup  E_{srcf}  \cup  E_{dstf}  \}$ the set of graph edges,
    \item $V_{flows} = \{ f_1, ... , f_n \}$: the set of $n$ nodes $f_i$ representing flows in the network,
    \item $V_{hosts} = \{ h_1, ... , h_m \}$: the set of $m$ nodes $h_i$ representing hosts in the network, including attackers and victims,
    \item $E_{fsrc} = \{ e = (u, fsrc , v) \And u \in V_{flows} \And v \in V_{hosts} \}$: the set of edges joining each flow u with its source host v,
    \item $E_{fdst} = \{  e = (u, fdst, v) \And u \in V_{flows}  \And v \in V_{hosts}  \}$: the set of edges joining each flow u with its destination v,
    \item $E_{srcf} = \{  e = (u, srcf , v) \And u \in V_{hosts} \And  v \in V_{flows}  \}$: the set of edges joining each host u with a flow v that it sends,
    \item $E_{dstf} = \{  e = (u, dstf , v) \And u \in V_{hosts} \And  v \in V_{flows}  \}$: the set of edges joining a host u to a flow v that it receives.
\end{itemize}

\subsection{Node's State}
In this paragraph, node's states are defined. Unlike existing approaches, a node's state in our proposed approach  contains multiple features obtained from different contributors. This paves the way for better expressiveness and optimizes the message passing process. \\
A node's state is defined as follows:
\begin{itemize}
    \item For hosts: the state $X_{h_i}$ of a host $h_i$ is presented by: $X_{h_i} = [S_{h_i} , D_{h_i}]$ where $S_{h_i}$ and $D_{h_i}$ represent source's and destination's states of $h_i$,respectively
    \item For flows: the state $X_{f_i}$ of a flow $f_i$ is defined as follows: $X_{f_i} = [S_{f_i},  X_{f_i},  D_{f_i} ]$ where:
    \begin{itemize}
        \item $X_{f_i}$: The features vector of the flow i,
        \item $S_{f_i}$: The features vector extracted from $S_i$,
        \item $D_{f_i}$: The features vector extracted from $D_i$,
    \end{itemize}

\end{itemize}

\subsection{Graphic Illustration}
After being formally defined, the proposed graph representation is presented with a graphic illustration to be more clarified. This subsection is dedicated for this aim. 
\begin{figure}[H]
    \centering
    \includegraphics[scale=0.85]{figures/graph_exp.png}
    \captionsetup{font=large}
    \caption{Graph Representation}
        \label{fig:graph_exp}
\end{figure}

The figure \ref{fig:graph_exp} demonstrates a graph that we notice G = (V, E) where: 
\begin{itemize}
    \item $V = V_{flows}  \cup  V_{hosts}$ and  $E = { E_{fsrc}  \cup  E_{fdst}  \cup  E_{fsrc}  \cup  E_{fdst} }$,
    \item $V_{flows} = \{ f_1, f_2, f_3, f_4, f_5, f_6, f_7, f_8 \}$ the set of 8 nodes representing flows in the network,
    \item $V_{hosts} = \{ h_1, h_2, h_3, h_4, h_5, h_6 \}$ the set of 6 nodes representing hosts in the network,
    \item $E_{fsrc}, E_{fdst}, E_{srcf} \text{ and } E_{dstf} $ are the sets of edges respectively of types fsrc, fdst, srcf and dstf.
\end{itemize}
The legend in the figure \ref{fig:graph_exp} indicates the corresponding color of each type of edges. 
According to the communications, G demonstrates that:
\begin{itemize}
    \item the host H1 sends the flow F1 to H2 and F2 and F3 to H4,
    \item H3 sends F4 and F5 to H2 and receives from it F6 and
    \item H5 sends F6 and F7 to H6, and receives from it F8.
\end{itemize}
According to the hidden states of the nodes, the figure \ref{fig:graph_exp} demonstrates the states of H1 and F1 as examples. 
\section{\textcolor{white}{.}Message \textcolor{white}{.}passing}
In the process of message passing, the following are considered:
\begin{itemize}
    \item  $m_{v, u, e} $= a message m is sent  from u to v through the edge e,
    % it will be used then
    \item $M_{v,t} = \text{AGGREGATION}({m_{v, u, e} \forall e \in  \mathcal{N}_t (v)})$: Aggregated message from all nodes u connected with edges of type t,
    % \item  $ M_{v,T} = combine({ M_{v,t} for t in T}) $ from all type t in T
    \item  $x_v = \text{UPDATE}(x_v , M_{v,t})$: UPDATE function updates only the sub-state $x_v$ of the node v.

\end{itemize}

 Initially, the state $X_{h_i}$ of a host $h_i$ is empty. All states are updated by performing the \textbf{flows\_to\_hosts message-passing}, happening from flows to sources through \textbf{fsrc} edges and  to destinations through \textbf{fdst} edges (as shown in Fig\ref{fig:approche_cg}).\\
Then, the second message-passing step is \textbf{hosts\_to\_flows} is performed between hosts and flows. The $f_i$'s state $X_{f_i}$ is updated by passing messages from $S_f$ and $D_f$ respectively through \textbf{srcf} and \textbf{dstf} edges. As a result, the final state  $z_{f_i}$ is computed by combining source's and destination's features with flow's features.\\
The algorithm \ref{approach_algo} proposes an implementation of the message-passing and update steps.\\
\begin{algorithm}[H]
\caption{Proposed message-passing and updating algorithm}\label{approach_algo}
\SetAlgoNlRelativeSize{-1}
\SetNlSty{textbf}{(}{)}
\SetAlgoNlRelativeSize{1}
\begin{algorithmic}
\REQUIRE{\textcolor{white}{.}G(V, E): \textcolor{white}{.}A \textcolor{white}{.}graph\textcolor{white}{.} G with\textcolor{white}{.} a\textcolor{white}{.} set\textcolor{white}{.} of\textcolor{white}{.} nodes\textcolor{white}{.} V\textcolor{white}{.} and\textcolor{white}{.} a\textcolor{white}{.} set\textcolor{white}{.} of\textcolor{white}{.} edges\textcolor{white}{.} E, X: Flows features matrix}
\KwResult{Z: The set of flows final states $z_i$}
\tcp{Initialising flows states}
\FOR{each $i \text{ in } V_{flows}$}
\STATE $S_{f_{i}}\leftarrow  []$
\STATE $D_{f_{i}}\leftarrow  []$
\STATE $X_{f_{i}}\leftarrow  X_i$
\ENDFOR\\
\tcp{Initialising hosts states}
\FOR{each $h_i \text{ in } V_{hosts}$}
\STATE $S_{h_{i}}\leftarrow  []$
\STATE $D_{h_{i}}\leftarrow  []$
\ENDFOR\\
\tcp{Update hosts representation}
\FOR{each $h_i \text{ in } V_{hosts}$}
\STATE $S_{h_{i}}\leftarrow  \text{UPDATE}(S_{h_{i}}, M_{h_i,fsrc})$
\STATE $D_{h_{i}}\leftarrow  \text{UPDATE}(D_{h_{i}}, M_{h_i,fdst})$
\ENDFOR
\tcp{Update flows representation}
\FOR{each $f_i \text{ in } V_{flows}$}
\STATE $S_{f_{i}}\leftarrow  \text{UPDATE}(S_{f_{i}}, M_{h_i,srcf})$
\STATE $D_{f_{i}}\leftarrow  \text{UPDATE}(D_{f_{i}}, M_{h_i,dstf})$
\STATE $z_{f_i} \leftarrow S_{f_{i}} \text{ || } X_{f_{i}} \text{ || } D_{f_{i}} $
\ENDFOR
\end{algorithmic}
\end{algorithm}

\subsection{Readout function}

Given the final flow nodes' representation, the GNN executes a readout phase \ref{readout} as seen in \ref{fig:app_fig}. The set of flows representation is forwarded into a trainable readout function \textbf{READOUT(·)} that produces the model output. 
In other words, this function is intended to map the final hidden-state of flow to it's corresponding label. This function predicts the label $y_{f_i}$ of a a node $f_i \in V_f$ thought its final embedding  $z_{f_i}$ as follows:

\begin{equation}
    y_{f_i} = \text{READOUT} (z_{f_i} ) \in \{0, 1\}
\end{equation}

\begin{figure}[H]
    \centering
    \includegraphics[scale=0.45]{figures/approche_cg.png}
    \captionsetup{font=large}
    \caption{The proposed message-passing mechanism}
    \label{fig:approche_cg}
\end{figure}


\section{Approach Discussion}
\section{Conclusion}
The proposed GNN model was described in this section throuht a detailed presentation for the proposed graph encoding and message passing mechanisme. In the next section, experimental evaluation of the proposed model will be presented and discussed.




